\documentclass{book}

\begin{document}
\title{How to solve it by computer: Python edition}
\maketitle

\chapter{Introduction To Computer Problem-Solving}
\section{Introduction}
Computer problem-solving can be summed up in one word- it is demanding! It is an intricate process requiring much thought, careful planning, logical precision, persistence, and attention to detail. At the same time it can be a challenging, exciting, and satisfying experience with considerable room for personal creativity and expression. If computer problem-solving is approached in this spirit then the chances of success are greatly amplified. In the discussion which follows in this introductory chapter we will attempt to lay the foundations for our study of computer problem-solving.
\subsection{Programs and algorithms}
The vehicle for the computer solution to a problem is a set of explicit and unambiguous instructions expressed in a programming language. This set of instructions is called a program. A program may also be thought of as an algorithm expressed in a programming language. An algorithm therefore corresponds to a solution to a problem that is independent of any programming language.\par
	To obtain the computer solution to a problem once we have to program we usually have to supply the program with input or data. The program then takes this input and manipulates it according to its instructions and eventually produces an output which represents the computer solution to the problem. The realization of the computer output is but the last step in a very long chain of events that have led up to the computer solution to the problem.\par

\end{document}