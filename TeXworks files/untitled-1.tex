\documentclass{book}

\begin{document}
\title{How to solve it by computer: Python edition}
\maketitle


\chapter*{CONTENTS}
\chapter{Introduction To Computer Problem-Solving}
\section{Introduction}
Computer problem-solving can be summed up in one word- it is \textit{demanding}! It is an intricate process requiring much thought, careful planning, logical precision, persistence, and attention to detail. At the same time it can be a challenging, exciting, and satisfying experience with considerable room for personal creativity and expression. If computer problem-solving is approached in this spirit then the chances of success are greatly amplified. In the discussion which follows in this introductory chapter we will attempt to lay the foundations for our study of computer problem-solving.
\subsection{Programs and algorithms}
The vehicle for the computer solution to a problem is a set of explicit and unambiguous instructions expressed in a programming language. This set of instructions is called a \textit{program}. A program may also be thought of as an algorithm expressed in a programming language. An \textit{algorithm} therefore corresponds to a solution to a problem that is \textit{independent} of any programming language.\par
	To obtain the computer solution to a problem once we have to program we usually have to supply the program with \textit{input} or data. The program then takes this input and manipulates it according to its instructions and eventually produces an \textit{output} which represents the computer solution to the problem. The realization of the computer output is but the last step in a very long chain of events that have led up to the computer solution to the problem.\par
	Our goal in this work is to study in depth the process of algorithm design with particular emphasis on the problem-solving aspects of the task. There are many definitions of an algorithm. The following definition is appropriate in computing science. An \textit{algorithm} consists of a set of explicit and unambiguous finite steps which, when carried out for a given set of initial conditions, produce the corresponding output and terminate in a finite time.

\subsection{Requirements for solving problems by computer}
	From time to time in our everyday activities, we employ algorithms to solve problems. For example, to look up someone's telephone number in a telephone directory we need to employ an algorithm. Tasks such as this are usually performed automatically without any thought to the complex under-lying mechanism needed to effectively conduct the search. It therefore comes as somewhat of a surprise to us when developing computer algorithms that the solution must be specified with such logical precision and in such detail. After studying even a small sample of computer problems it soon becomes obvious that the conscious \textit{depth of understanding} needed to design effective computer algorithms is far greater than we are likely to encounter in almost any other problem-solving situation.\par
	Let us reflect for a moment on the telephone directory look-up problem. A telephone directory quite often contains hundreds of thousands of names and telephone numbers yet we have little trouble finding the desired telephone number we are seeking. We might therefore ask why do we have so little difficulty with a problem of seemingly great size? The answer is simple. We quite naturally take advantage of the order in the directory to quickly eliminate large sections of the list and home in on the desired name and number. We would never contemplate looking up the telephone number of J. R. Nash by starting at page 1 and examining each name in turn until we finally come to Nash's name and telephone number. Nor are we likely to contemplate looking up the name of the person whose number is 2987533. To conduct such a search, there is no way in which we can take advantage of the order in the directory and so we are faced with the prospect of doing a number-by-number search starting at page 1. If, on the other hand, we had a list of telephone numbers and names ordered by telephone number rather than name, the task would be straightforward. What these examples serve to emphasize is the important influence of the data organization on the performance of algorithms. Only when a data structure is symbiotically linked with an algorithm can we expect high performance. Before considering these and other aspects of algorithm design we need to address the topic of problem-solving in some detail. 

\section{THE PROBLEM-SOLVING ASPECT}
	It is widely recognized that problem-solving is a creative process which largely defies systematization and mechanization. This may not sound very encouraging to the would-be problem-solver. To balance this, most people, during their schooling, acquire at least a modest set of problem-solving skills which they may or may not be aware of.\par
	Even if one is not naturally skilled at problem-solving there are a number of steps that can be taken to raise the level of one's performance. It is not implied or intended that the suggestions in what follows are in any way a recipe for problem-solving. The plain fact of the matter is that there is no universal method. Different strategies appear to work for different people.\par
	Within this context, then, where can we begin to say anything useful about computer problem-solving? We must start from the premise that computer problem-solving is about understanding. 

\subsection{Problem definition phase}
	Success in solving any problem is only possible after we have made the effort to come to terms with or understand the problem at hand. We cannot hope to make useful progress in solving a problem until we fully understand what it is we are trying to solve. This preliminary investigation may be thought of as the \textit{problem definition phase}. In other words, what we must do during this phase is work out \textit{what must be done} rather \textit{than how to do it}. That is, we must try to extract from the problem statement (which is often quite imprecise and maybe even ambiguous) a set of precisely defined tasks. Inexperienced problem-solvers too often gallop ahead with how they are going to solve the problem only to find that they are either solving the wrong problem or they are solving just a very special case of what is actually required. In short, a lot of care should be taken in working out precisely what must be done. The development of algorithms for finding the square root (algorithm 3.1) and the greatest common divisor (algorithm 3.3) are good illustrations of how important it is to carefully define the problem. Then, from the definitions, we are led in a natural way to algorithm designs for these two problems. 

\subsection{Getting started on a problem}
	There are many ways to solve most problems and also many solutions to most problems. This situation does not make the job of problem-solving easy. When confronted with many possible lines of attack it is usually difficult to recognize quickly which paths are likely to be fruitless and which may be productive.\par
	Perhaps the more common situation for people just starting to come to grips with the computer solution to problems is that they just do not have any idea where to start on the problem, even after the problem definition phase. When confronted with this situation, what can we do? A block often occurs at this point because people become concerned with details of the implementation \textit{before} they have completely understood or worked out an implementation-independent solution. The best advice here is not to be too concerned about detail. That can come later when the complexity of the problem as a whole has been brought under control. The old computer proverb' which says "the sooner you start coding your program the longer it is going to take" is usually painfully true.\par

\subsection{The use of specific examples}
A useful strategy when we are stuck is to use some props or heuristics (i.e. rules of thumb) to try to get a start with the problem. An approach that often allows us to make a start on a problem is to pick a specific example of the general problem we wish to solve and try to work out the mechanism that will allow us to solve this particular problem (e.g. if you want to find the maximum in a set of numbers, choose a particular set of numbers and work out the mechanism for finding the maximum in this set—see for example algorithm 4.3). It is usually much easier to work out the details of a solution to a specific problem because the relationship between the mechanism and the particular problem is more clearly defined. Furthermore, a specific problem often forces us to focus on details that are not so apparent when the problem is considered abstractly. Geometrical or schematic diagrams rep-resenting certain aspects of the problem can be usefully employed in many instances (see, for example, algorithm 3.3).\par
	This approach of focusing on a particular problem can often give us the foothold we need for making a start on the solution to the general problem. The method should, however, not be abused. It is very easy to fall into the trap of thinking that the solution to a specific problem or a specific class-of problems is also a solution to the general problem. Sometimes this happens but we should always be very wary of making such an assumption.\par
	Ideally, the specifications for our particular problem need to be examined very carefully to try to establish whether or not the proposed algorithm can meet those requirements. If the full specifications are difficult to formulate sometimes a well-chosen set of test cases can give us a degree of confidence in the generality of our solution. However, nothing less than a complete proof of correctness of our algorithm is entirely satisfactory. We will discuss this matter in more detail a little later.









\end{document}